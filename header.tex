%%%%%%%%%%%%%%%%%%%%%%%%%%%%%%%%%%%%%%%%%%%%%%%%%%%%%%%%%%%%%%%%%%%%%%%%%%%%%%%%
%%%%%%%%%%%%%%%%%%   Vorlage für eine Abschlussarbeit   %%%%%%%%%%%%%%%%%%%%%%%%
%%%%%%%%%%%%%%%%%%%%%%%%%%%%%%%%%%%%%%%%%%%%%%%%%%%%%%%%%%%%%%%%%%%%%%%%%%%%%%%%

% Erstellt von Maximilian Nöthe, <maximilian.noethe@tu-dortmund.de>
% ausgelegt für lualatex und Biblatex mit biber

% Kompilieren mit
% latexmk --lualatex --output-directory=build thesis.tex
% oder einfach mit:
% make

\documentclass[
  tucolor,       % remove for less green,
  BCOR=12mm,     % 12mm binding corrections, adjust to fit your binding
  parskip=half,  % new paragraphs start with half line vertical space
  open=any,      % chapters start on both odd and even pages
  cleardoublepage=plain,  % no header/footer on blank pages
]{tudothesis}


% Warning, if another latex run is needed
\usepackage[aux]{rerunfilecheck}

% just list chapters and sections in the toc, not subsections or smaller
\setcounter{tocdepth}{1}

%------------------------------------------------------------------------------
%------------------------------ Fonts, Unicode, Language ----------------------
%------------------------------------------------------------------------------
\usepackage{fontspec}
\defaultfontfeatures{Ligatures=TeX}  % -- becomes en-dash etc.
\setmonofont{Fira Mono}[Scale=MatchLowercase]

% load all used languages
% and set the main language of this thesis
% use this if this thesis is written in German:
% \usepackage[english, ngerman]{babel}
% use this if this thesis is written in English:
\usepackage[ngerman, american]{babel}

% intelligent quotation marks, language and nesting sensitive
\usepackage[autostyle]{csquotes}

% microtypographical features, makes the text look nicer on the small scale
\usepackage{microtype}

%------------------------------------------------------------------------------
%------------------------ Math Packages and settings --------------------------
%------------------------------------------------------------------------------

\usepackage{amsmath}
\usepackage{amssymb}
\usepackage{mathtools}

% Enable Unicode-Math and follow the ISO-Standards for typesetting math
\usepackage[
  math-style=ISO,
  bold-style=ISO,
  sans-style=italic,
  nabla=upright,
  partial=upright,
]{unicode-math}
\setmathfont{Latin Modern Math}
\setmathfont{XITS Math}[range={scr, bfscr}]
\setmathfont{XITS Math}[range={cal, bfcal}, StylisticSet=1]

% nice, small fracs for the text with \sfrac{}{}
\usepackage{xfrac}


%------------------------------------------------------------------------------
%---------------------------- Numbers and Units -------------------------------
%------------------------------------------------------------------------------

\usepackage[
  separate-uncertainty=true,
  per-mode=symbol-or-fraction,
]{siunitx}
\sisetup{math-micro=\text{µ},text-micro=µ}
% automatically choose the right locale
\addto\extrasngerman{\sisetup{locale = DE}}
\addto\extrasenglish{\sisetup{locale = UK}}

%------------------------------------------------------------------------------
%-------------------------------- tables  -------------------------------------
%------------------------------------------------------------------------------

\usepackage{booktabs}       % \toprule, \midrule, \bottomrule, etc

%------------------------------------------------------------------------------
%-------------------------------- graphics -------------------------------------
%------------------------------------------------------------------------------

\usepackage{graphicx, eso-pic}
% currently broken
% \usepackage{grffile}
% \usepackage{titlepic}

% allow figures to be placed in the running text by default:
\usepackage{scrhack}
\usepackage{float}
\floatplacement{figure}{htbp}
\floatplacement{table}{htbp}

% keep figures and tables in the section
\usepackage[section, below]{placeins}

\usepackage{tikz}
\usetikzlibrary{overlay-beamer-styles,calc,tikzmark,decorations.pathreplacing}
\tikzset{fontscale/.style = {font=\relsize{#1}}}
\usepackage{feynman-tikz}


%------------------------------------------------------------------------------
%---------------------- customize list environments ---------------------------
%------------------------------------------------------------------------------

\usepackage{enumitem}

%------------------------------------------------------------------------------
%------------------------------ bibliography ----------------------------------
%------------------------------------------------------------------------------

\usepackage[
  backend=biber,
  giveninits=true, % abbreviate all first names for consistency
  urldate=iso,
  seconds=true,    % needed for urldate=ISO, silences warning
]{biblatex}
\addbibresource{references.bib}  % the bib file to use
% \DefineBibliographyStrings{german}{andothers = {{et\,al\adddot}}}  % replace u.a. with et al.


\usepackage{pdfpages}
\usepackage{geometry}

%------------------------------------------------------------------------------
%------------------------------ Glossaries ------------------------------------
%------------------------------------------------------------------------------
\usepackage[xindy, toc]{glossaries}
\setacronymstyle{long-short}
\makeglossaries



%------------------------------------------------------------------------------

\usepackage{todonotes}


% Last packages: do not change order or insert new packages after these
\usepackage[pdfusetitle, unicode, hidelinks]{hyperref}
\usepackage{bookmark}
\usepackage[shortcuts]{extdash}