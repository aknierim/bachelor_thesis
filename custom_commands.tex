% Defining a new coordinate system for the page, see
% https://tex.stackexchange.com/questions/89588/positioning-relative-to-page-in-tikz/89592#89592
\makeatletter
\def\parsecomma#1,#2\endparsecomma{\def\page@x{#1}\def\page@y{#2}}
\tikzdeclarecoordinatesystem{page}{
    \parsecomma#1\endparsecomma
    \pgfpointanchor{current page}{north east}
    % Save the upper right corner
    \pgf@xc=\pgf@x%
    \pgf@yc=\pgf@y%
    % save the lower left corner
    \pgfpointanchor{current page}{south west}
    \pgf@xb=\pgf@x%
    \pgf@yb=\pgf@y%
    % Transform to the correct placement
    \pgfmathparse{(\pgf@xc-\pgf@xb)/2.*\page@x+(\pgf@xc+\pgf@xb)/2.}
    \expandafter\pgf@x\expandafter=\pgfmathresult pt
    \pgfmathparse{(\pgf@yc-\pgf@yb)/2.*\page@y+(\pgf@yc+\pgf@yb)/2.}
    \expandafter\pgf@y\expandafter=\pgfmathresult pt
}
\makeatother

% reference for the page coordinate system
% --------------------------
% |(-1,1)    (0,1)    (1,1)|
% |                        |
% |(-1,0)    (0,0)    (1,0)|
% |                        |
% |(-1,-1)   (0,-1)  (1,-1)|
% --------------------------

% ------------------------------------------------------------------------------

% some custom commands
\newcommand{\colorrule}[3][black]{\textcolor{#1}{\rule{#2}{#3}}}
\newcommand{\wip}[2][red]{\textcolor{#1}{\textbf{#2}}}

% renewed commands
\renewcommand*{\glstextformat}[1]{\textcolor{black!70}{\lining #1}}

\renewcommand*{\chapterformat}{\fontsize{48}{48}\color{tugreen}\selectfont\thechapter\autodot\enskip}
\renewcommand*{\sectionformat}{\makebox[0pt][r]{\thesection\autodot\enskip}}
\renewcommand*{\subsectionformat}{\makebox[0pt][r]{\thesubsection\autodot\enskip}}

% metrics
\DeclareMathOperator{\tp}{tp}
\DeclareMathOperator{\fp}{fp}
\DeclareMathOperator{\fn}{fn}
\DeclareMathOperator{\tn}{tn}
\DeclareMathOperator{\recall}{recall}
\DeclareMathOperator{\precision}{precision}
\DeclareMathOperator{\tpr}{tpr}
\DeclareMathOperator{\fpr}{fpr}
\DeclareMathOperator{\tnr}{tnr}
\DeclareMathOperator{\fnr}{fnr}
\DeclareMathOperator{\eff}{Eff}
\NewDocumentCommand \roc {} {\gls{roc}}
\NewDocumentCommand \auc {} {\gls{auc}}

% common abbreviations
\NewDocumentCommand \eg {} {e.\,g.\ }
\NewDocumentCommand \ie {} {i.\,e.\ }
\NewDocumentCommand \wrt {} {w.\,r.\,t.\ }
\NewDocumentCommand \etal {} {et.\,al.\ }

% custom units
\DeclareSIUnit\sigma{σ}
\DeclareSIUnit\pe{\symup{p}.\symup{e}.}

% custom physical quantities
\NewDocumentCommand \Eref {o} {\ensuremath{{E_\text{ref}\IfValueT{#1}{^{#1}}}}}
\NewDocumentCommand \Emax {o} {\ensuremath{{E_\text{max}\IfValueT{#1}{^{#1}}}}}
\NewDocumentCommand \Emin {o} {\ensuremath{{E_\text{min}\IfValueT{#1}{^{#1}}}}}
\NewDocumentCommand \Eest {o} {\ensuremath{{E_\text{est}\IfValueT{#1}{^{#1}}}}}
\NewDocumentCommand \Etrue {o} {\ensuremath{{E_\text{true}\IfValueT{#1}{^{#1}}}}}
\NewDocumentCommand \tobs {o} {\ensuremath{{t_\text{obs}\IfValueT{#1}{^{#1}}}}}
\NewDocumentCommand \Rmax {o} {\ensuremath{{R_\text{max}\IfValueT{#1}{^{#1}}}}}
\NewDocumentCommand \Aeff {o} {\ensuremath{{A_\text{eff}\IfValueT{#1}{^{#1}}}}}
\NewDocumentCommand \pii {o} {\ensuremath{\symup{π}}}

% software names
\NewDocumentCommand \ctapipe {} {\texttt{ctapipe}}
\NewDocumentCommand \numpy {} {\texttt{numpy}}
\NewDocumentCommand \astropy {} {\texttt{astropy}}
\NewDocumentCommand \matplotlib {} {\texttt{matplotlib}}
\NewDocumentCommand \pandas {} {\texttt{pandas}}
\NewDocumentCommand \sklearn {} {\texttt{scikit-learn}}
\NewDocumentCommand \pyirf {} {\texttt{pyirf}}

% maths operators
\let\textd\d
\RenewDocumentCommand \d {m} {\TextOrMath{\textd{#1}}{\mathinner{\symup{d}#1}}}

% often named glossary terms
\NewDocumentCommand \cta {} {\gls{cta}}

% cleaning algorithms
\NewDocumentCommand \tailcuts {} {\texttt{TailcutsImageCleaner}}
\NewDocumentCommand \mars {} {\texttt{MARSImageCleaner}}
\NewDocumentCommand \fact {} {\texttt{FACTImageCleaner}}
\NewDocumentCommand \tcc {} {\texttt{TimeConstrainedImageCleaner}}

% data levels
\NewDocumentCommand \rzero {} {\textbf{R}\(\mathbf{0}\)}
\NewDocumentCommand \rone {} {\textbf{R}\(\mathbf{1}\)}
\NewDocumentCommand \dlz {} {\textbf{DL}\(\mathbf{0}\)}
\NewDocumentCommand \dloa {} {\textbf{DL}\(\mathbf{1}\)\textbf{a}}
\NewDocumentCommand \dlob {} {\textbf{DL}\(\mathbf{1}\)\textbf{b}}
\NewDocumentCommand \dlt {} {\textbf{DL}\(\mathbf{2}\)}
\NewDocumentCommand \dlo {} {\textbf{DL}\(\mathbf{1}\)}



% define a macro \Autoref to allow multiple references to be passed to \autoref
% (from https://tex.stackexchange.com/a/183682)
\makeatletter
\newcommand\Autoref[1]{\@first@ref#1,@}
\def\@throw@dot#1.#2@{#1}% discard everything after the dot
\def\@set@refname#1{%    % set \@refname to autoefname+s using \getrefbykeydefault
    \edef\@tmp{\getrefbykeydefault{#1}{anchor}{}}%
    \xdef\@tmp{\expandafter\@throw@dot\@tmp.@}%
    \ltx@IfUndefined{\@tmp autorefnameplural}%
         {\def\@refname{\@nameuse{\@tmp autorefname}s}}%
         {\def\@refname{\@nameuse{\@tmp autorefnameplural}}}%
}
\def\@first@ref#1,#2{%
  \ifx#2@\autoref{#1}\let\@nextref\@gobble% only one ref, revert to normal \autoref
  \else%
    \@set@refname{#1}%  set \@refname to autoref name
    \@refname~\ref{#1}% add autoefname and first reference
    \let\@nextref\@next@ref% push processing to \@next@ref
  \fi%
  \@nextref#2%
}
\def\@next@ref#1,#2{%
   \ifx#2@ and~\ref{#1}\let\@nextref\@gobble% at end: print and+\ref and stop
   \else, \ref{#1}% print  ,+\ref and continue
   \fi%
   \@nextref#2%
}
\makeatother