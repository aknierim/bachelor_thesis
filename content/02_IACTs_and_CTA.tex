\chapter{IACTs and the Cherenkov Telescope Array}
\glsreset{iact}

Most modern gamma-ray observations are performed with either space-based experiments or with
\glspl{iact}, which are ground-based telescopes or arrays of telescopes that use the Cherenkov light
emitted by \glspl{eas} in the atmosphere. In the following sections I will introduce \glspl{iact} and
the \gls{cta} and explain the mechanisms that make it possible to observe gamma rays with these types
of experiments.


\section{Imaging Air Cherenkov Telescopes}

Because of their ground-based setup, \glspl{iact} are taking advantage of the Earth's atmosphere to get a
larger effective area than any space-based instrument. This is especially helpful for energies above
\todo{confirm values and add source}\SI{100}{\giga\eV}, where the gamma-ray flux is low compared to lower energies
that have higher fluxes. This means, that space-based experiments with their small effective area
will see fewer high-energy events compared to ground-based experiments. The cosmic ray flux in
\todo{add figure}\autoref{fig:flux} shows this well: The flux decreases rapidly with higher energies, with only one
high-energy photon per day and square meter reaching Earth from the Crab Nebula \cite{noethe_thesis}.

% \begin{figure}
%     \centering
%     \input{build/cosmic_rays.pgf}
%     \caption{The cosmic ray flux as a function of energy. The flux is given in \si{\per\second\per\square\meter\per\steradian}.}
%     \label{fig:flux}
% \end{figure}
% \todo{placeholder, maybe use a different plot?}

For high-energy gamma rays, Earth's atmosphere is opaque, so ground-based experiments rely on
cascades of sub-particles in so-called \glspl{eas}. For gamma-ray astronomy, the electromagnetic
component of these \glspl{eas}, shown in \autoref{fig:heitler_model}, is more \todo{wording}relevant
than the hadronic component. When a gamma ray interacts with Earth's atmosphere, it decays into an
electron and a positron via pair production. These charged particles then emit more photons via
bremsstrahlung. These photons, again, produce more charged particles, which in turn emit more
photons. This process continues until any of these processes reaches energies below a certain
threshold, \ie \SI{1022}{\kilo\eV} for the pair production process. For these processes to happen,
the particles have to be in the fields of the atoms of the Earth's atmosphere. As the particles
travel at speeds faster than light in the medium of the atmosphere, they emit Cherenkov light at
a fixed angle \(\theta\) with respect to the refraction index \(n\) of the atmosphere and the
factor \(\beta = \sfrac{v}{\symup{c}}\). The angle can be determined trigonometrically as
\begin{equation}
    \cos\theta = \frac{1}{\beta n},
\end{equation}
as can be seen in \autoref{fig:cherenkov}.



\begin{figure}
    \begin{subfigure}[t]{0.45\textwidth}
        \centering
        \includegraphics[width=\textwidth]{graphics/heitler_model.pdf}
        \caption{Simplified Heitler model of the electromagnetic component of an extensive air shower.
        A gamma ray induces an electron and a positron via pair production that then emit more photons
        via bremsstrahlung.}
        \label{fig:heitler_model}
    \end{subfigure}
    \hfill
    \begin{subfigure}[t]{0.45\textwidth}
        \centering
        \includegraphics[width=\textwidth]{graphics/cherenkov_radiation.pdf}
        \caption{Cherenkov radiation (outward-pointing arrows) from a charged particle traveling at
        uniform velocity, faster than the speed of light in the medium. The dashed line shows the
        path of the particle over time.}
        \label{fig:cherenkov}
    \end{subfigure}
    \caption{The electromagnetic component of an extensive air shower (see \autoref{fig:heitler_model})
    is induced by gamma rays that then decay into electrons and positrons via pair production.
    These charged particles will emit more gamma rays via bremsstrahlung, which again will produce
    new electrons and positrons. This will continue until \eg the energy of the gamma rays is below
    the threshold of \SI{1022}{\kilo\eV} for pair production. Another process is the production of
    Cherenkov light from the electrons and positrons, which is shown in \autoref{fig:cherenkov}.
    The Cherenkov light is the result of the charged particles traveling faster than the speed of light
    in the atmosphere, thus emitting photons in a fixed angle \(\theta\) \wrt the refraction index
    \(n\) of the medium and the factor \(\beta\).}
    \label{fig:cherenkov_heitler}
\end{figure}

Cherenkov light emitted by \glspl{eas} can then be collected by an IACT's mirrors and detected by
its camera within a timeframe of an order of nanoseconds. The resulting image will show the shape
of the impact of the air shower and can be used to determine the shower's primary particles' properties,
and the type and reconstruct its origin. As the hadronic component of \glspl{eas} produce
electromagnetic subshowers, \glspl{iact} have a dominant hadronic background, which has to be
\todo{maybe write something about ML}separated from the gamma ray-induced showers.


\section{The Cherenkov Telescope Array}

The \cta{} is a new generation of \glspl{iact} that will consist of two sites,
one of which will be built at the \gls{orm} on the Canarian island of La Palma while the other
will be built in the southern hemisphere at the \glspl{eso} Paranal Observatory in the Atacama desert
of northern Chile. In this work, I will focus on the northern array, also called \cta{} north,
on La Palma.

The \cta{} consists of three types of telescopes, each being sensitive to a different energy range:
\begin{description}
    \item [] \textbf{\gls{lst}}: The largest telescopes in the \cta{} have primary deflector diameter
    of \SI{23}{\meter} with a field of view of \SI{4.3}{\deg} and are sensitive to energy
    ranges from \(\SI{20}{\giga\eV}\) to \(\SI{150}{\giga\eV}\) \todo{add source}\cite{cta_specs}.
    \item [] \textbf{\gls{mst}}: The \glspl{mst} have a primary deflector diameter of \SI{11.5}{\meter}
    with a field of view of \SI{7.7}{\deg} for their NectarCam and are sensitive to energy ranges
    from \(\SI{150}{\giga\eV}\) to \(\SI{5}{\tera\eV}\).
    \item [] \textbf{\gls{sst}}: Being the smallest telescope type in the \cta{}, the \glspl{sst}
    have a primary deflector diameter of \SI{4.3}{\meter}, a secondary deflector diameter of
    \SI{1.8}{\meter} with a field of view of \SI{10.5}{\deg} and are sensitive to energy ranges
    from \(\SI{5}{\tera\eV}\) to \(\SI{300}{\tera\eV}\).
\end{description}

\begin{figure}
    \centering
    \includegraphics[width=\textwidth]{graphics/cta_layout.png}
    \caption{Full layout of the \cta{} north and south sites. \cta{} south would feature
    \(\num{70}\) \glspl{sst}, \(\num{25}\) \glspl{mst} and \(\num{4}\) \glspl{lst}, while
    \cta{} north would feature \(\num{15}\) \glspl{mst} and \(\num{4}\) \glspl{lst} in its
    full configuration \cite{cta_layout}.}
    \label{fig:cta_layout}
\end{figure}

Since the southern hemisphere has more galactic high-energy sources than the northern hemisphere,
only \cta{} south will feature the \glspl{sst} along the other two telescope types. \cta{} north
will feature the \glspl{mst} and \glspl{lst}. \todo{decide on final figure}\autoref{fig:cta_layout} shows the
planned full layout of both sites of the \cta{} where \cta{} south is displayed with its \(\num{70}\)
\glspl{sst}, \(\num{25}\) \glspl{mst} and the \(\num{4}\) \glspl{lst} in the center. \cta{} north
will feature \(\num{15}\) \glspl{mst} and \(\num{4}\) \glspl{lst}. Due to the simulation data used,
this work will focus on the reduced layout, also called \todo{check correct name} Prod5b alpha
configuration, for the northern site, consisting of \(\num{9}\) \glspl{mst} and \(\num{4}\) \glspl{lst}.