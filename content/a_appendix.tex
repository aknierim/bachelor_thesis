% \renewcommand{\thechapter}{\Alph{chapter}}
\chapter{Appendix}
\label{ap:appendix}

\section{Configurations for \ctapipe{}}
\label{ap:config_files}

The listing below shows the configuration file used for preprocessing the datasets, \ie{} before the
application of the different cleaning settings.
\begin{spacing}{0.5}
    \begin{mdframed}[backgroundcolor=codebg, hidealllines=true, leftmargin=0cm,rightmargin=0cm, skipabove=0pt, innerleftmargin=0,innerrightmargin=0,]
    \lstinputlisting[basicstyle=\lstsansserif, language=yaml]{./configs/preprocessing.yml}
    \end{mdframed}
\end{spacing}

The listings below show the contents of the default configuration files used with \ctapipe{}
for each cleaning algorithm respectively.

\begin{description}
    \item \textbf{\tailcuts{}}:\medskip
    \begin{spacing}{0.5}
        \begin{mdframed}[backgroundcolor=codebg, hidealllines=true, leftmargin=0cm,rightmargin=0cm, skipabove=0pt, innerleftmargin=0,innerrightmargin=0,]
        \lstinputlisting[basicstyle=\lstsansserif, language=yaml]{./configs/tailcuts_clean_config.yml}
        \end{mdframed}
    \end{spacing}

    \item \textbf{\mars{}}:\medskip
    \begin{spacing}{0.5}
        \begin{mdframed}[backgroundcolor=codebg, hidealllines=true, leftmargin=0cm,rightmargin=0cm, skipabove=0pt, innerleftmargin=0,innerrightmargin=0,]
        \lstinputlisting[basicstyle=\lstsansserif, language=yaml]{./configs/mars_cleaning_1st_pass_config.yml}
        \end{mdframed}
    \end{spacing}

    \item \textbf{\fact{}}:\medskip
    \begin{spacing}{0.5}
        \begin{mdframed}[backgroundcolor=codebg, hidealllines=true, leftmargin=0cm,rightmargin=0cm, skipabove=0pt, innerleftmargin=0,innerrightmargin=0,]
        \lstinputlisting[basicstyle=\lstsansserif, language=yaml]{./configs/fact_image_cleaning_config.yml}
        \end{mdframed}
    \end{spacing}

    \item \textbf{\tcc{}}:\medskip
    \begin{spacing}{0.5}
        \begin{mdframed}[backgroundcolor=codebg, hidealllines=true, leftmargin=0cm,rightmargin=0cm, skipabove=0pt, innerleftmargin=0,innerrightmargin=0,]
        \lstinputlisting[basicstyle=\lstsansserif, language=yaml]{./configs/time_constrained_cleaning_config.yml}
        \end{mdframed}
    \end{spacing}
\end{description}

The following listing shows the contents of the \texttt{prod5b\_lapalma\_alpha.yml} configuration file,
which is used to set the allowed telescope IDs for \ctapipe{}\texttt{-process}.
\begin{spacing}{0.5}
    \begin{mdframed}[backgroundcolor=codebg, hidealllines=true, leftmargin=0cm,rightmargin=0cm, skipabove=0pt, innerleftmargin=0,innerrightmargin=0,]
    \lstinputlisting[basicstyle=\lstsansserif, language=yaml]{./configs/prod5b_lapalma_alpha.yml}
    \end{mdframed}
\end{spacing}

Since the comparison of the cleaning algorithms also required differentiating between \glspl{mst}
and \glspl{lst}, the following two listings show the contents of the \texttt{prod5b\_lapalma\_mst.yml} and
\texttt{prod5b\_lapalma\_lst.yml} configuration files, respectively:
\begin{description}
    \item \textbf{\glspl{mst}}:\medskip
    \begin{spacing}{0.5}
        \begin{mdframed}[backgroundcolor=codebg, hidealllines=true, leftmargin=0cm,rightmargin=0cm, skipabove=0pt, innerleftmargin=0,innerrightmargin=0,]
        \lstinputlisting[basicstyle=\lstsansserif, language=yaml]{./configs/prod5b_lapalma_mst.yml}
        \end{mdframed}
    \end{spacing}

    \item \textbf{\glspl{lst}}:\medskip
    \begin{spacing}{0.5}
        \begin{mdframed}[backgroundcolor=codebg, hidealllines=true, leftmargin=0cm,rightmargin=0cm, skipabove=0pt, innerleftmargin=0,innerrightmargin=0,]
        \lstinputlisting[basicstyle=\lstsansserif, language=yaml]{./configs/prod5b_lapalma_lst.yml}
        \end{mdframed}
    \end{spacing}
\end{description}


\section{Software used}

This work was written and built with \LaTeX{} and Lua\TeX{} from \TeX Live 2021 on both Windows 10 and Ubuntu 18.04.
Also, I relied heavily on the python programming language, of which the most important libraries used
for this work are listed below.
\begin{itemize}
    \item \numpy{}~\cite{numpy}
    \item \pandas{}~\cite{pandas}
    \item \matplotlib{}~\cite{matplotlib}
    \item \astropy{}~\cite{astropy1, astropy2}
    \item \pyirf{}~\cite{pyirf}
    \item \sklearn{}~\cite{scikit-learn}
\end{itemize}

For the processing of the datasets, I used the development version of \texttt{ctapipe}, more specifically, version
\texttt{0.15.1.dev166+gf26107f}. A complete listing of all used python libraries can be found below:
\begin{spacing}{0.5}
    \begin{mdframed}[backgroundcolor=codebg, hidealllines=true, leftmargin=0cm,rightmargin=0cm, skipabove=0pt, innerleftmargin=0,innerrightmargin=0,]
    \lstinputlisting[basicstyle=\footnotesize\lstsansserif, multicols=2, language=yaml]{./configs/env.yml}
    \end{mdframed}
\end{spacing}
