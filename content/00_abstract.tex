\thispagestyle{plain}

\section*{Abstract}
In recent years, gamma astronomy has become a vastly studied field of astrophysics.
Since then many \gls{iact} experiments have been producing many insights into sources of
high-energy gamma-rays. Continuing this trend, the \gls{cta}, a next-generation \gls{iact},
will provide new data at some of the highest energies of the gamma-ray spectrum.

In this thesis, I take a look at the four cleaning algorithms implemented in the low-level data processing software \ctapipe{},
currently developed for \gls{cta}. Furthermore, I will determine optimized hyperparameters for each algorithm
with the aim of improving the performance of the cleaning algorithms. This is achieved with a grid search
over a range of reasonable hyperparameters. The data used in this work is diffuse gamma \gls{mc} data for
the \glspl{mst} at the northern site of \cta{} on the Canarian island of La Palma.

The \tcc{} algorithm in \ctapipe{} was improved over its default implementation \wrt to the angular
resolution, while the other algorithms were only slightly improved or at least returned similar results.
In terms of metrics, the \tcc{} algorithm only improved slightly compared to its default, while \fact{}
performed worse than its default implementation. Overall, however, \fact{} performed better than the other
algorithms.


\section*{Kurzfassung}
\begin{otherlanguage}{ngerman}
In den vergangenen Jahrzehnten hat sich die Gammaastronomie zu einem viel erforschten Bereich der
Astrophysik entwickelt. Seitdem haben bildgebende, atmosphärische Tscherenkow-Teleskope (\gls{iact})
viele neue Erkenntnisse im Bereich der hochenergetischen Gammastrahlen und deren Quellen geliefert.
Mit \gls{cta}, einem \gls{iact} der nächsten Generation, werden neue Daten in einigen der höchsten
Energiebereiche der Gammastrahlen erwartet.

In dieser Arbeit schaue ich mir die vier Bildreinigungsalgorithmen an, die in der sich zur Zeit für \cta{}
in Entwicklung befindlichen Software \ctapipe{} implementiert sind. Zudem bestimme ich optimierte
Hyperparameter für jeden Algorithmus mit dem Ziel, die Leitung der Bildreinigungsalgorithmen zu verbessern.
Dies wird mittels einer Grid-Search über eine Auswahl von Hyperparametern durchgeführt. Die Daten, die zu diesem
Zweck verwendet werden, bestehen aus diffusen Gammadaten, die mittels einer \gls{mc}-Simulation für die
\glspl{mst} des nördlichen Standorts \gls{cta}s auf der Kanarischen Insel La Palma erzeugt wurden.

Die Ergebnisse zeigen, dass eine Verbesserung des \tcc{}-Algorithmus über dessen Standardimplementierung
in Bezug auf die Winkelauflösung erreicht werden konnte, während die anderen Algorithmen nur ein wenig
verbessert wurden oder etwa gleiche Ergebnisse lieferten. In Bezug auf die Metriken, hat sich gezeigt,
dass der \tcc{}-Algorithmus nur ein wenig verbessert wurde, während der \fact{}-Algorithmus schlechter war als
seine Standardimplementierung. Insgesamt lieferte \fact{} allerdings bessere Ergebnisse als die anderen Algorithmen.


\end{otherlanguage}
\glsresetall
% \listoftodos
