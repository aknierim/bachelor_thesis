\thispagestyle{plain}

\section*{Abstract}
In recent years, gamma astronomy has become a vastly studied field of astrophysics.
Since then, many \gls{iact} experiments have been producing insights into sources of
high-energy gamma-rays. Continuing this trend, the \gls{cta}, a next-generation array of \glspl{iact},
will provide new data targeting some of the highest energies of the gamma-ray spectrum.

In this thesis, I consider the four cleaning algorithms implemented in the low-level data processing software \ctapipe{},
currently developed for \gls{cta}. Furthermore, I will determine optimized hyperparameters for each algorithm
with the aim of improving their performance. A grid search achieves this objective
over a range of reasonable hyperparameters. The data used in this work is diffuse gamma \gls{mc} simulation data for
the \glspl{mst} at the northern site of \cta{} on the Canarian island of La Palma.

The hyperparameters determined for the algorithms in \ctapipe{} improve their performance over
their default parameters significantly \wrt the angular resolution. This is especially true for the
time-based \tcc{} algorithm, which shows the most improvement \wrt the metrics, closely followed by the
non-time-based \mars{} algorithm. Overall, \fact{} and \mars{} perform the best out of all cleaning algorithms in this work,
with the latter being notable for its consistency in terms of the angular resolution over the whole energy range.


\section*{Kurzfassung}
\begin{otherlanguage}{ngerman}
In den vergangenen Jahrzehnten hat sich die Gammaastronomie zu einem viel erforschten Bereich der
Astrophysik entwickelt. Seitdem haben bildgebende, atmosphärische Tscherenkow-Teleskope (\glspl{iact})
zahlreiche neue Erkenntnisse im Bereich der hochenergetischen Gammastrahlen und deren Quellen geliefert.
Mit \gls{cta}, einem Observatorium bestehend aus dutzenden \glspl{iact} der nächsten Generation, werden neue Daten in einigen der höchsten
Energiebereiche der Gammastrahlen erwartet.

In dieser Arbeit betrachte ich die vier Bildreinigungsalgorithmen, die in der Software \ctapipe{} implementiert
sind, welche sich zur Zeit für \cta{} in Entwicklung befindet. Zudem bestimme ich optimierte
Hyperparameter für jeden Algorithmus mit dem Ziel, die Leistung der Bildreinigungsalgorithmen zu verbessern.
Dies wird mittels einer Grid-Search über eine Auswahl von Hyperparametern durchgeführt. Die zu diesem
Zweck verwendeten Daten bestehen aus diffusen Gammadaten, die mittels einer \gls{mc}-Simulation für die
\glspl{mst} des nördlichen Standorts \gls{cta}s auf der Kanarischen Insel La Palma erzeugt wurden.

Die Ergebnisse zeigen, dass die in dieser Arbeit gefundenen Hyperparameter die Leistung aller vier
Algorithmen in Bezug auf die Winkelauflösung verbessern. Dies gilt insbesondere für den \tcc{}-Algorithmus,
der sich auch in Bezug auf die Metriken am meisten verbessert, gefolgt vom \mars{}-Algorithmus.
Insgesamt arbeiten \fact{} und \mars{} von allen vier Algorithmen in dieser Arbeit am besten, wobei
letzterer insbesondere bei der Winkelauflösung über den gesamten Energiebereich hinweg
konstant gute Ergebnisse liefert.
% Die Ergebnisse zeigen, dass die in dieser Arbeit gefundenen Hyperparameter die Leistung des \tcc{}-Algorithmus
% in Bezug auf die Winkelauflösung verbesserten, während die Leistungen der anderen Algorithmen nur wenig
% verbessert wurden oder gleiche Ergebnisse lieferten, wie mit den Standardparametern.
% In Bezug auf die Metriken konnten die gefundenen Hyperparameter bei \tcc{} eine leichte Verbesserung
% bewirken, während die Parameter bei \fact{} sogar eine etwas schlechtere Leistung bewirkten.
% Insgesamt lieferte \fact{} allerdings bessere Ergebnisse als die anderen Algorithmen.
\end{otherlanguage}
\glsresetall%
% \listoftodos
