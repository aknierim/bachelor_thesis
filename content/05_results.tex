\chapter{Results}
\label{ch:results}

In this chapter, the results of the analysis are presented. First, I present the results of the
efficiency analysis in \autoref{sec:efficiency_angres}. The initial tests of parameter combinations
are based on a small dataset consisting of \(\num{20}\) runs, \ie \(\num{12668}\) events. This is
due to the number of parameter combinations that are tested, as more runs would increase the
processing time immensely. Furthermore, I show the results of the angular resolution for a combined
metric with the efficiency. Then, in \autoref{sec:metrics}, the metrics of each resulting combination
of hyperparameters are plotted. In \autoref{sec:performance}, the performance of each cleaning
algorithm compared to the default settings is shown. Finally, a comparison of the different
cleaning algorithms is provided in \autoref{sec:comparison}.


\section{Analysis of the efficiency and the angular resolution}
\label{sec:efficiency_angres}

To narrow down possible candidates for the optimal hyperparameters, I first analyzed the efficiency
of the different cleaning algorithms. The efficiency is determined by the number of events that are
reconstructed after cleaning. For this work I chose \(\num{20}\) intervals within \(\num{0}\) and
\(\num{1}\) with a step size of \(\num{0.05}\). The events are binned per energy and the mean efficiency over all
energy bins is then calculated as the mean of \autoref{eq:efficiency}. For each interval only those datasets are selected, where the mean efficiency
lies between the lower and upper bound of the interval. Then the minimum angular resolution is
determined for each interval. The parameters of these datasets are then selected to be the optimal
parameters for each cleaning algorithm. This not only allows for a comparison of the cleaners but also
a decision on a trade-off between the efficiency and the angular resolution, namely having a better
angular resolution, but a lower efficiency or a higher efficiency but a higher and therefore worse
angular resolution. The results for the mean efficiency are listed in \autoref{tab:efficiency} and
the results for the mean angular resolution in \autoref{tab:angres}.

As one can see, not all cleaning algorithms have valid values for each interval, peaking at a maximum
of around \(\SIrange{45}{50}{\percent}\) of successfully reconstructed events. This is because not
all events are stereo events, \ie events, where two or more telescopes were triggered, which is a
requirement for the reconstruction.
The remaining events are therefore mono events and do not contribute to either the efficiency or the
angular resolution. The efficiency would be higher, of course, for a full-array analysis, but that would
mean also including \glspl{lst} data\footnote{which was left out due to time limitations, see \autoref{ch:conclusions}.},
which for this work would not help find the optimal parameters, since it is better to analyze the telescope
types separately. The reason for the latter is, that optimizing the telescopes by type would lead to
better results for the hyperparameters.
\begin{table}
    \centering
    \caption{The results of the analysis for the mean efficiency of each cleaning algorithm taken over all
    energy bins. The efficiency is calculated as the ratio of the number of reconstructed events $n_{\mathrm{reco}}$ and the number
    of total events $n_{\mathrm{total}}$. The table lists the lower and upper limits of each efficiency
    interval. The efficiency is then calculated as the mean over the whole energy range of the dataset and
    each listed efficiency is the one where the mean angular resolution is minimal for the given
    interval. Notice how not all cleaning algorithms have valid results for all efficiency intervals.}
    \label{tab:efficiency}
    \rowcolors{0}{white!92!black}{}
    \begin{tabular}{S[table-format=1.2] S[table-format=1.2] S[table-format=1.3] S[table-format=1.3] S[table-format=1.3] S[table-format=1.3]}
        \hiderowcolors
        & & \multicolumn{4}{c}{Mean Efficiency} \\
        {$\eff_{\mathrm{lower}}$} & {$\eff_{\mathrm{upper}}$} & {Tailcuts} & {MARS} & {FACT} & {TCC} \\
        \addlinespace[0.5em]
        \showrowcolors
        % \input{build/efficiency.txt}
        0.00 & 0.05 &       &       & 0.034 &       \\
        0.05 & 0.10 &       &       & 0.051 &       \\
        0.10 & 0.15 &       &       & 0.117 &       \\
        0.15 & 0.20 &       &       & 0.163 &       \\
        0.20 & 0.25 &       &       & 0.218 & 0.211 \\
        0.25 & 0.30 & 0.263 & 0.265 & 0.287 & 0.273 \\
        0.30 & 0.35 & 0.313 & 0.315 & 0.321 & 0.316 \\
        0.35 & 0.40 & 0.362 & 0.364 & 0.369 & 0.390 \\
        0.40 & 0.45 & 0.402 & 0.403 & 0.426 & 0.426 \\
        0.45 & 0.50 & 0.451 & 0.463 & 0.466 & 0.455 \\
        0.50 & 0.55 &       & 0.501 &       &       \\
    \end{tabular}
\end{table}

\begin{table}
    \centering
    \caption{The results of the analysis for the mean angular resolution of each cleaning algorithm.
    The table lists the lower and upper limits of each efficiency interval. The angular resolution listed
    is the minimum mean angular resolution of the respective efficiency interval. The corresponding efficiency
    values are listed in \autoref{tab:efficiency}. Notice how not all cleaning algorithms have valid results
    for all efficiency intervals.}
    \label{tab:angres}
    \rowcolors{0}{white!92!black}{}
    \begin{tabular}{S[table-format=1.2] S[table-format=1.2] S[table-format=1.3] S[table-format=1.3] S[table-format=1.3] S[table-format=1.3]}
        \hiderowcolors
        & & \multicolumn{4}{c}{Mean Angular Resolution\(\;/\; \si{\degree}\)} \\
        {$\eff_{\mathrm{lower}}$} & {$\eff_{\mathrm{upper}}$} & {Tailcuts} & {MARS} & {FACT} & {TCC} \\
        \addlinespace[0.5em]
        \showrowcolors
        % \input{build/angular_resolution.tex}
        0.00 & 0.05 &       &       & 0.244 &       \\
        0.05 & 0.10 &       &       & 0.297 &       \\
        0.10 & 0.15 &       &       & 0.420 &       \\
        0.15 & 0.20 &       &       & 0.428 &       \\
        0.20 & 0.25 &       &       & 0.477 & 0.413 \\
        0.25 & 0.30 & 0.358 & 0.291 & 0.415 & 0.396 \\
        0.30 & 0.35 & 0.308 & 0.282 & 0.367 & 0.340 \\
        0.35 & 0.40 & 0.334 & 0.332 & 0.366 & 0.386 \\
        0.40 & 0.45 & 0.357 & 0.343 & 0.365 & 0.383 \\
        0.45 & 0.50 & 0.395 & 0.395 & 0.404 & 0.390 \\
        0.50 & 0.55 &       & 1.301 &       &       \\
    \end{tabular}
\end{table}
The results listed in the tables show that there is a clear trade-off in choosing between efficiency and angular resolution.
As such, for further comparison of the cleaning algorithms, the corresponding datasets for the efficiency and
the angular resolution are plotted in \autoref{fig:efficiency_angres} for the intervals
\([\num{0.25}, \num{0.30}]\) and \([\num{0.45}, \num{0.50}]\). The reason for this is that these are
the minimum and maximum intervals \wrt the efficiency, where all cleaners have valid values.
Furthermore, the mean angular resolution is plotted against the efficiency in \autoref{fig:ar_vs_eff}.
\vspace{-0.2cm}
\begin{figure}
    \centering
    \begin{subfigure}{0.48\textwidth}
        \centering
        \includegraphics[width=0.85\textwidth]{plots/ar_aeff/AR_Aeff_MST_0.25_0.30.pdf}
    \end{subfigure}
    \hfill
    \begin{subfigure}{0.48\textwidth}
        \centering
        \includegraphics[width=0.85\textwidth]{plots/ar_aeff/AR_Aeff_MST_0.45_0.50.pdf}
    \end{subfigure}
    \caption{Mean angular resolution and efficiency for the MST simulation binned per energy. Notice
    the decrease in the scattering of the mean angular resolution at medium to
    medium-high energies at higher efficiencies in the plots on the right.\vspace{-0.5cm}}
    \label{fig:efficiency_angres}
\end{figure}
\begin{figure}[!htbp]
    \centering
    \includegraphics[width=0.75\textwidth]{build/ar_vs_eff.pdf}
    \caption{Angular resolution plotted against the efficiency between \(\num{0.2}\) and \(\num{0.5}\).}
    \label{fig:ar_vs_eff}
\end{figure}


\section{Metrics of the cleaning algorithms}
\label{sec:metrics}
\glsreset{tp}\glsreset{fp}\glsreset{fn}\glsreset{tn}
Concerning the metrics, I first analyzed the parameter combinations of all settings found in \autoref{sec:efficiency_angres}. First,
the number of the \gls{tp}, \gls{fp}, \gls{fn} and \gls{tn} values is calculated per event. Then,
those values are summed up for each cleaning algorithm and the metrics are calculated as shown in
\autoref{tab:metrics}.

The metrics for each parameter combination are first compared for each cleaning algorithm respectively,
to determine the best setting within a set of parameter combinations. The results are then plotted in
\Autoref{fig:metrics_tail, fig:metrics_mars, fig:metrics_fact, fig:metrics_tcc}
and are shown with specific IDs they were given when the datasets were processed with \ctapipe.
This improves readability as opposed to writing out the full settings. The hyperparameters for the best performing
IDs of all algorithms are listed in \autoref{tab:best_parameters}.

While all settings perform well w.\,r.\,t., for example, the \gls{tnr}, one can see from the results, that for all cleaning
algorithms---except for \mars{}---, the best parameter combination is the one where the \gls{tpr}, the \gls{acc} and the \gls{ba}
are the highest. These settings are the ones that correspond to the highest efficiency. While this is true
for \tailcuts{}, \fact{} and \tcc{}, the metrics for \mars{} are an exception. The reason for that is, that when
looking at the corresponding mean angular resolution for the highest performing setting (ID~10), the value is
\(\SI{1.301}{\degree}\) (see \autoref{tab:angres}) and therefore an order of magnitude worse than the other algorithms. For better
comparison, I, therefore, choose the second highest performing setting (ID~15) with a mean angular
resolution of \(\SI{0.395}{\degree}\) to be the best candidate for further analysis.
The values for each respective metric and cleaning algorithm are listed in
\Autoref{tab:metrics_tail, tab:metrics_mars, tab:metrics_fact, tab:metrics_tcc} respectively, while
a comparison of the cleaners against each other is shown in \autoref{sec:comparison}.
\begin{table}
    \centering
    \caption{Results for the metrics of \tailcuts{}. One can see, that the best results are obtained
    for the setting with ID~47.}
    \label{tab:metrics_tail}
    \rowcolors{0}{white!92!black}{}
    \adjustbox{varwidth=\linewidth,scale=0.9}{%
    \begin{tabular}{r S[table-format=1.4] S[table-format=1.4] S[table-format=1.4] S[table-format=1.4] S[table-format=1.4] S[table-format=1.4] S[table-format=1.4]}
        \hiderowcolors
        {ID} & \acrshort{tpr} & \acrshort{tnr} & \acrshort{fnr} & \acrshort{acc} & \acrshort{ba} \\
        \addlinespace[0.5em]
        \showrowcolors
        % \input{build/metrics_tail.txt} %% kinda broken atm
        118 & 0.0972 & 1.0000 & 0.9028 & 0.9256 & 0.5486 \\
         51 & 0.1231 & 1.0000 & 0.8769 & 0.9256 & 0.5615 \\
        140 & 0.1460 & 1.0000 & 0.8540 & 0.9256 & 0.5730 \\
         81 & 0.1854 & 1.0000 & 0.8146 & 0.9288 & 0.5927 \\
         47 & 0.2268 & 1.0000 & 0.7732 & 0.9332 & 0.6134 \\
    \end{tabular}}
\end{table}

\begin{figure}
    \centering
    \includegraphics[width=0.95\textwidth]{build/metrics_tailcuts.pdf}
    \caption{Metrics for \tailcuts{}. It can be seen, that the cleaning setting with ID 47 performs
    best in terms of the true positive rate, accuracy and balanced accuracy, making it the best
    setting of the five parameter combinations.}
    \label{fig:metrics_tail}
\end{figure}

\begin{table}
    \centering
    \caption{Results for the metrics of \mars{}. The last row (ID~10) shows a significant increase in the \gls{tpr}
    and therefore decrease in \gls{fnr} and another significant increase in the \gls{ba} value. Since
    this specific parameter combination also corresponds to a mean angular resolution an order higher
    than the rest, I chose to select the second highest performing setting (ID~15) for further analysis
    instead.}
    \label{tab:metrics_mars}
    \rowcolors{0}{white!92!black}{}
    \begin{tabular}{r S[table-format=1.4] S[table-format=1.4] S[table-format=1.4] S[table-format=1.4] S[table-format=1.4] S[table-format=1.4] S[table-format=1.4]}
        \hiderowcolors
        ID & \acrshort{tpr} & \acrshort{tnr} & \acrshort{fnr} & \acrshort{acc} & \acrshort{ba} \\
        \addlinespace[0.5em]
        \showrowcolors
        \input{build/metrics_mars.txt}\\
    \end{tabular}
\end{table}

\begin{figure}
    \centering
    \includegraphics[width=\textwidth]{build/metrics_mars.pdf}
    \caption{Metrics for \mars{}. While ID~10 arguably performs best at first glance---albeit with a not
    insignificant decrease in the \gls{ppv} value---, I chose ID~15
    for further analysis, as its mean angular resolution is an order of magnitude better.}
    \label{fig:metrics_mars}
\end{figure}

\begin{table}
    \centering
    \caption{Results for the metrics of \fact{}. One can see, that the best results are obtained
    for the settings with ID~503.}
    \label{tab:metrics_fact}
    \rowcolors{0}{white!92!black}{}
    \begin{tabular}{r S[table-format=1.4] S[table-format=1.4] S[table-format=1.4] S[table-format=1.4] S[table-format=1.4] S[table-format=1.4] S[table-format=1.4]}
        \hiderowcolors
        ID & \acrshort{tpr} & \acrshort{tnr} & \acrshort{fnr} & \acrshort{acc} & \acrshort{ba} \\
        \addlinespace[0.5em]
        \showrowcolors
        \input{build/metrics_fact.txt}\\
    \end{tabular}
\end{table}

\begin{figure}
    \centering
    \includegraphics[width=\textwidth]{build/metrics_fact.pdf}
    \caption{Metrics for \fact{}. The cleaning setting with ID 503 performs
    best in terms of the number of true positives, accuracy and balanced accuracy, making it the best
    setting of the ten parameter combinations.}
    \label{fig:metrics_fact}
\end{figure}

\begin{table}
    \centering
    \caption{Results for the metrics of \tcc{}. The best results are obtained
    for the setting with ID~807.}
    \label{tab:metrics_tcc}
    \rowcolors{0}{white!92!black}{}
    \begin{tabular}{r S[table-format=1.4] S[table-format=1.4] S[table-format=1.4] S[table-format=1.4] S[table-format=1.4] S[table-format=1.4] S[table-format=1.4] }
        \hiderowcolors
        ID & \acrshort{tpr} & \acrshort{tnr} & \acrshort{fnr} & \acrshort{acc} & \acrshort{ba} \\
        \addlinespace[0.5em]
        \showrowcolors
        \input{build/metrics_tcc.txt}\\
    \end{tabular}
\end{table}

\begin{figure}
    \centering
    \includegraphics[width=\textwidth]{build/metrics_tcc.pdf}
    \caption{Metrics for the \tcc{}. The cleaning setting with ID 807 performs
    best in terms of the number of true positives, accuracy and balanced accuracy, making it the best
    setting of the six parameter combinations.}
    \label{fig:metrics_tcc}
\end{figure}

\begin{table}
    \centering
    \caption{Best performing IDs and the corresponding hyperparameters for each respective cleaner.
    Note, that, as discussed above, the best performing ID \wrt
    the metrics for \mars{} is, in fact, not the best, but the second best. This is because the mean angular resolution is
    an order of magnitude better for ID~15 than the one for ID~10. For better readability, the names
    of the algorithms are shortened. Listed are the core threshold \(Q_c\), the boundary threshold \(Q_b\),
    the minimum number of neighbors and where applicable the time limit \(t\) and core and boundary time limits
    \(t_c\) and \(t_b\).}
    \label{tab:best_parameters}
    \rowcolors{0}{white!92!black}{}
    \begin{tabular}{c S[table-format=2.0] S[table-format=1.3]
        S[table-format=1.3] S[table-format=1.0] S[table-format=2.1] S[table-format=2.1] S[table-format=2.1]}
        \hiderowcolors
        {Cleaning Algorithm} & {ID} & {\(Q_c \;/\; \si{\pe}\)} & {\(Q_b \;/\; \si{\pe}\)} & {Min. Neigh.} &
        {\(t \;/\; \si{\nano\second}\)} & {\(t_c \;/\; \si{\nano\second}\)} & {\(t_b \;/\; \si{\nano\second}\)} \\
        \addlinespace[0.5em]
        \showrowcolors
        Tailcuts &  47 & 6.200 & 4.650 & 2 &      &      &      \\
        MARS     &  15 & 6.700 & 4.467 & 1 &      &      &      \\
        FACT     & 503 & 6.700 & 3.350 & 1 & 12.0 &      &      \\
        TCC      & 807 & 6.700 & 4.467 & 1 &      & 15.0 & 12.0 \\
    \end{tabular}
\end{table}

\section{Performance compared to the default settings}
\label{sec:performance}

Now, with a setting selected for each cleaner, it is reasonable to compare
the performance to the default settings, that are implemented in the \ctapipe{} source code
(see \autoref{tab:hyperparameters}). First, the relative angular resolution
\begin{equation}
    \theta_\text{rel} = \frac{\theta_{\SI{68}{\percent}}}{\theta_{\SI{68}{\percent},\,\text{base}}}
\end{equation}
is computed for the default (base) settings and the selected settings for each cleaner.
The results are shown in \autoref{fig:rel_angres} for the same efficiency intervals
as in \autoref{fig:efficiency_angres}. One can see, that especially for the interval
\([0.45, 0.5)\) and at medium to high energies, \tcc{} performs fairly well compared to the default settings.
The other algorithms perform only slightly better than the default settings at a performance gain
of about \(\SI{10}{\percent}\).

Furthermore, the metrics of the hyperparameters determined in this work can be compared to the
metrics of the default settings. The results are shown in \autoref{fig:metrics_comparison} with the
values of the metrics being listed in \autoref{tab:metrics_default}. The values for the selected
settings in this work are listed in \autoref{tab:metrics_all}.
\fact{} performs worse than its default setting \wrt the metrics, especially in terms of
\gls{tpr} and \gls{ba}. This might be due to the higher core and boundary thresholds compared to the
default values, as lower values would allow selecting more pixels. \tcc{}, however, performs
significantly better than the default settings, especially in terms of \gls{tpr} and \gls{ba}, but
also slightly better in terms of the \gls{acc} metric.
\begin{figure}
    \centering
    \includegraphics[width=\textwidth]{build/metrics_baseline.pdf}
    \caption{Metrics for the default (baseline) settings compared to the settings determined in this work.
    One can see that an improvement was reached for \tcc, especially in terms of \gls{tpr} and \gls{ba}, while
    \fact{} performs worse than its default implementation.}
    \label{fig:metrics_comparison}
\end{figure}
\begin{figure}
    \centering
    \begin{subfigure}[t]{0.45\textwidth}
        \centering
        \includegraphics[width=\textwidth]{plots/ar_aeff/Rel_AR_0.25_0.30_base.pdf}
    \end{subfigure}
    \hfill
    \begin{subfigure}[t]{0.45\textwidth}
        \centering
        \includegraphics[width=\textwidth]{plots/ar_aeff/Rel_AR_0.45_0.50_base.pdf}
    \end{subfigure}
    \caption{Relative angular resolution for the default settings and the selected settings for each cleaner.
    \tcc{} outperforms the default values with the new settings, while the other algorithms improve only slightly.}
    \label{fig:rel_angres}
\end{figure}

\begin{table}
    \centering
    \caption{Metrics for the default (baseline) settings. The metrics show that \fact{}
    performs well even in its default settings and also better than with the hyperparameters
    determined in this work. \tcc{}, however, performs better with the hyperparameters listed in
    \autoref{tab:best_parameters} \wrt the metrics.}
    \label{tab:metrics_default}
    \rowcolors{0}{white!92!black}{}
    \begin{tabular}{c S[table-format=1.4] S[table-format=1.4] S[table-format=1.4]
        S[table-format=1.4] S[table-format=1.4] S[table-format=1.4] S[table-format=1.4]}
        \hiderowcolors
        {Cleaning Algorithm} & {\acrshort{tpr}} & {\acrshort{fpr}} & {\acrshort{tnr}} &
        {\acrshort{fnr}} & {\acrshort{ppv}} & {\acrshort{acc}} & {\acrshort{ba}} \\
        \addlinespace[0.5em]
        \showrowcolors
        Tailcuts & 0.2316 & 0.0000 & 1.0000 & 0.7684 & 0.9971 & 0.9358 & 0.6158 \\
        MARS     & 0.2346 & 0.0000 & 1.0000 & 0.7654 & 0.9970 & 0.9358 & 0.6173 \\
        FACT     & 0.3261 & 0.0002 & 0.9998 & 0.6739 & 0.9864 & 0.9429 & 0.6629 \\
        TCC      & 0.1790 & 0.0000 & 1.0000 & 0.8210 & 0.9995 & 0.9299 & 0.5895 \\

    \end{tabular}
\end{table}

\section{Comparison of the cleaning algorithms}
\label{sec:comparison}

As of writing this thesis, the cleaning algorithms discussed in this thesis are the only ones implemented
in the \ctapipe{} source code. With the determined hyperparameters, I decided to compare the performance of all
four cleaners. I once again decided to use the settings selected in \autoref{sec:metrics} for each algorithm
and look at the metrics. The metrics of all four algorithms are shown in \autoref{fig:metrics_all}.
The corresponding values are listed in \autoref{tab:metrics_all}.

One can see, that \fact{} performs best in terms of \gls{tpr} and \gls{ba}, while \mars{} and
\tailcuts{} perform well on \gls{acc} and \gls{ppv} respectively. Overall, however, \fact{} seems
to be a good choice for cleaning since it performs reasonably well even on \gls{acc} and \gls{ppv}.


\begin{table}
    \centering
    \caption{Metrics for the selected settings of each cleaning algorithm. Out of these
    four algorithms, \fact{} performs best in terms of \gls{tpr} and \gls{ba}, while \mars{} and
    \tailcuts{} perform well on \gls{acc} and \gls{ppv} respectively. \fact{}, however, performs
    reasonably well in the scope of the resulting metrics of this work and is, therefore, a good
    overall choice for cleaning.}
    \label{tab:metrics_all}
    \rowcolors{0}{white!92!black}{}
    \begin{tabular}{c S[table-format=1.4] S[table-format=1.4] S[table-format=1.4]
        S[table-format=1.4] S[table-format=1.4] S[table-format=1.4] S[table-format=1.4]}
        \hiderowcolors
        {Cleaning Algorithm} & {\acrshort{tpr}} & {\acrshort{fpr}} & {\acrshort{tnr}} &
        {\acrshort{fnr}} & {\acrshort{ppv}} & {\acrshort{acc}} & {\acrshort{ba}} \\
        \addlinespace[0.5em]
        \showrowcolors
        Tailcuts & 0.2268 & 0.0000 & 1.0000 & 0.7732 & 0.9994 & 0.9332 & 0.6134 \\
        MARS     & 0.2433 & 0.0000 & 1.0000 & 0.7567 & 0.9985 & 0.9380 & 0.6216 \\
        FACT     & 0.2655 & 0.0000 & 1.0000 & 0.7345 & 0.9968 & 0.9369 & 0.6327 \\
        TCC      & 0.2314 & 0.0000 & 1.0000 & 0.7686 & 0.9989 & 0.9364 & 0.6157 \\
    \end{tabular}
\end{table}

\begin{figure}
    \centering
    \includegraphics[width=\textwidth]{build/metrics_all.pdf}
    \caption{Bar plot visualizing the metrics for the selected settings of all four cleaning algorithms.
    \fact{} performs best out of all cleaning algorithms, followed by \mars{} and \tcc{}. The latter, however
    performs well \wrt the angular resolution, as shown in \autoref{fig:rel_angres} and can therefore be considered
    a viable alternative to \fact{}.}
    \label{fig:metrics_all}
\end{figure}