\documentclass{standalone}
\usepackage{fontspec}
\defaultfontfeatures{Ligatures=TeX}  % -- becomes en-dash etc.
\setmonofont{Fira Mono}[Scale=MatchLowercase]
\usepackage{tikz}
\usetikzlibrary{overlay-beamer-styles,calc,tikzmark,decorations.pathreplacing}
\tikzset{fontscale/.style = {font=\relsize{#1}}}

% new commands
\newcommand{\code}[2]{%
  \texttt{\textcolor{#1}{\detokenize{#2}}}%
}

\usepackage{xcolor}

% define your colors here
\xdefinecolor{bioblue}{HTML}{4c5b75}
\xdefinecolor{bredongreen}{HTML}{619645}
\xdefinecolor{honeywax}{HTML}{f9a825}
\xdefinecolor{protonred}{HTML}{a12b2b}
\xdefinecolor{electronblue}{HTML}{0099d1}
\xdefinecolor{gammagreen}{HTML}{0a9e74}

\xdefinecolor{tugreen}{RGB}{132, 184, 25}
\xdefinecolor{tulightgreen}{HTML}{99b560}
\xdefinecolor{darkmode}{HTML}{3a3d41}
\colorlet{tulight}{tugreen!20!white}
\colorlet{tudark}{tugreen!80!black}

\xdefinecolor{tuorange}{RGB}{227, 105, 19}
\xdefinecolor{tuyellow}{RGB}{242, 189, 0}
\xdefinecolor{tucitron}{RGB}{249, 219, 0}

\xdefinecolor{tublue}{RGB}{25, 132, 184}
\colorlet{tublight}{tublue!20!white}
\colorlet{tubdark}{tublue!60!black}

\xdefinecolor{yamlblue}{HTML}{a094f2}
\xdefinecolor{yamlgreen}{HTML}{00b300}
\xdefinecolor{yamlorange}{HTML}{d67c58}
\xdefinecolor{yamlpink}{HTML}{ff00ff}
\xdefinecolor{yamlyellow}{HTML}{ffd800}

\colorlet{lightgray}{darkgray!70!white}
\colorlet{lightergray}{darkgray!50!white}

\begin{document}

\feynmandiagram [small, vertical=a to b, tree layout, xscale=0.5, scale=1.5]  {
    a [particle=\(\gamma\)] -- [photon] b,
    c [] -- [fermion, edge label=\(e^{+}\)] b -- [fermion, edge label=\(e^{-}\),] d,

    c1 -- [fermion, edge label=\(e^{+}\)] c -- [photon, edge label=\(\gamma\)] c2,
    d1 -- [photon, photon, edge label=\(\gamma\)] d -- [fermion, edge label=\(e^{-}\)] d2,

    c11 [particle=\(e^{+}\)] -- [fermion] c1 -- [photon] c12 [particle=\(\gamma\)],
    d11 [particle=\(e^{+}\)]  -- [fermion] d1 -- [fermion] d12 [particle=\(e^{-}\)],
    c21 [particle=\(e^{+}\)] -- [fermion] c2 -- [fermion] c22 [particle=\(e^{-}\)],
    d21 [particle=\(\gamma\)]  -- [photon] d2 -- [fermion] d22 [particle=\(e^{-}\)],
};
\end{document}
